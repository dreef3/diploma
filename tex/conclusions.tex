\chapter*{Заключение}

\section*{Достигнутые результаты}

В ходе практической части работы удалось проанализировать и реализовать алгоритм построения условных планов, пользовательский интерфейс к нему, утилиты импорта описания предметной области и экспорта описания графа условного плана.

Еженедельные обсуждения теоретической части позволили обнаружить слабые места алгоритма планирования и найти пути их устранения. Итоговый продукт будет доработан в соответствии с планом работ, представленным ниже, для возможного дальнейшего коммерческого применения.

\section*{План дальнейшей работы}

Можно выделить следующие компоненты, в которых будет производится дальнейшая разработка:

\begin{itemize}
 \item Оптимизация алгоритма планирования для обработки случаев, рассмотренных в \S~\ref{sec:plantemplates}
 \item Повышение производительности и максимальных размеров описания задачи за счет кеширования частей плана. Исследование возможностей генерирования плана по мере развития диалога
 \item Реализация инструментария для нагрузочного тестирования, описанного в \S~\ref{subsec:highload}
 \item Разработка системы обучения с учителем для создания описаний предметных областей и задач на них без использования языка PDDL.
 \item Создание интерфейса пользователя в виде веб-приложения
\end{itemize}
